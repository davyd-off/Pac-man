\chapter{\label{ch:ch01}ГЛАВА 1. ТЕОРЕТИЧЕСКАЯ ЧАСТЬ.}

\section{\label{sec:ch01/sec01}Техническое задание.}
\begin{enumerate}
	\item Кроссплатформенное приложение, способное запускаться на операционных системах Windows и GNU/Linux.
	\item Написание игры Pac-Man на языке программирования Python3.
	\item Использование графической библиотеки PySDL2~\ref{pysdl2}.
	\item Возможность выбора карты или генерация карты в игре Pac-Man.
	\item Возможность выбора количества врагов в игре Pac-Man.
	\item Движение призраков не так важно, главное --- чтобы они преследовали пакмана.
	\item Наличие в игре кнопки <<Помощь>>, при нажатии на которую выводится информация об игре(например, её управлении), а также реферат об этой игре.
\end{enumerate}

\section{\label{sec:ch01/sec02}Графическая библиотека.}
Для создания кроссплатформенной программы на языке программирования Python3 была использована графическая библиотека PySDL2~\cite{pypiRUENpysdl2,docENpysdl2}.

\textbf{PySDL2}~\label{pysdl2} --- это чистая оболочка Python для библиотек SDL2, SDL2\_mixer, SDL2\_image, SDL2\_ttf и SDL2\_gfx. Вместо того, чтобы полагаться на код С, она использует
встроенный модуль \texttt{ctypes} для взаимодействия с SDL2 и предоставляет простые классы Python и оболочки для общих функций SDL2.

PySDL2 состоит из двух пакетов: \texttt{sdl2}, который представляет собой простую API-оболочку 1:1 вокруг API SDL2, и \texttt{sdl2.ext}, который предлагает расширенные функциональные возможности для \texttt{sdl2}.

Пакет \texttt{sdl2} реализован таким образом, что его можно легко интегрировать и развертывать с собственными программными проектами.

Модуль \texttt{sdl2.ext} предоставляет богатый набор модулей, классов и функций для создания игр и других приложений с использованием PySDL2.

Цель модуля \texttt{sdl2.ext} --- обернуть часто используемые части API SDL2 более дружественным и питоническим способом, уменьшая необходимость для разработчиков разбираться в тонкостях работы \texttt{ctypes} и делая создание программ PySDL2 более простым и увлекательным.
Кроме того, этот модуль предоставляет ряд классов шаблонов и служебных функций для работы с цветами, событиями ввода, файловыми ресурсами и т. д.


\section{\label{sec:ch01/sec03}Инструментарий.}
Для написания отчёта с помощью системы компьютерной верстки в \TeX была использована IDE TexStudio~\cite{texRUtexstud,texENtexstud}, а также дистрибутив Tex Live~\cite{texRUtexlive}.

Для написания кода программы была использована IDE Microsoft Visual Studio Code~\cite{vscodeEN}.

Для работы с изображениями, используемых в ходе разработки программы, был использован графический редактор Adobe Photoshop 2023~\cite{adobeRU}.

Для хранения проекта была выбрана система контроля версия GitHub~\cite{wikiRUGitHub}.

Проверка работоспособности и сборка программы выполнялась на системе:
	\begin{itemize}
		\item \textbf{OC}: \textit{Windows 10}
		\item \textbf{ЦП}: \textit{AMD FX-6300}
		\item \textbf{ОЗУ}: \textit{8gb}
		\item \textbf{Видеокарта}: \textit{NVIDIA GeForce GTX 1050}
	\end{itemize}
	

\section{\label{sec:ch01/sec04}Игра Pac-Man.}
\subsection{\label{subsec:ch01/sec04/subsec01}История игры.}
\textbf{Pac-Man} --- аркадная видеоигра, разработанная японской компанией Namco и вышедшая в 1980 году~\cite{pacmanRU}.

В конце 1970-х из-за успеха игры Space Invaders рынок видеоигр сосредоточился на создании аркадных шутемапов и привлекал к себе сугубо мужскую аудиторию. Сотрудник Namco Тору Иватани хотел сделать игру, которая понравилась бы всем, в особенности девушкам, и в качестве основной темы своей работы выбрал еду. В Японии созданная игра получила название Puck-Man, но при локализации в США компания Midway изменила название на Pac-Man, считая, что при оригинальном названии дети могли бы закрасить середину буквы «P», превратив её в «F», и сделать слово обсценным.

После выхода в Японии игра была принята хорошо, но не стала популярной. В Америке же аудитория была впечатлена отсутствием в аркаде насильственного мотива, что привлекло в том числе женскую аудиторию и помогло заработать лояльность родителей к видеоигре. Всё это обеспечило популярность Pac-Man среди людей разных возрастов и профессий, породив повышенный интерес к игре и сделав Пакмана первой звездой видеоигр. Такой успех вдохновил разработчиков на создание более разнообразных игр, в том числе с отсутствием насилия.

Благодаря популярности игра портировалась и переиздавалась на множество платформ, среди которых особую известность получила версия Pac-Man для Atari 2600. Из-за своего плохого качества она стала одним из символов кризиса видеоигр 1983 года. Pac-Man дала начало одноимённой серии игр, в которой вышло множество продолжений на различных платформах, включая аркадные автоматы, домашние игровые системы, компьютеры и мобильные телефоны. Помимо этого, вышло множество клонов, подражаний и нелегальных копий игры.

\subsection{\label{subsec:ch01/sec04/subsec02}Правила игры.}
Экран игры представляет собой лабиринт, коридоры которого заполнены точками.

Задача игрока --- управляя Пакманом, съесть все точки в лабиринте, избегая встречи с привидениями, которые гоняются за героем.

В начале каждого уровня призраки находятся в недоступной Пакману прямоугольной комнате в середине уровня, из которой они со временем освобождаются. Если привидение дотронется до Пакмана, то его жизнь теряется, призраки и Пакман возвращаются на исходную позицию, но при этом прогресс собранных точек сохраняется. Если при столкновении с призраком у Пакмана не осталось дополнительных жизней, то игра заканчивается. После съедения всех точек начинается новый уровень в том же лабиринте. По бокам лабиринта находятся два входа в один туннель, при вхождении в который Пакман и призраки выходят с другой стороны лабиринта.

Всего в лабиринте находятся 240 маленьких точек и 4 большие, известные как энерджайзеры . За съедение маленькой точки даётся 10 очков, а за съедение энерджайзера — 50. Таким образом, в общей сложности все точки в лабиринте стоят 2600 очков. При съедении Пакманом энерджайзера призраки в лабиринте на короткое время входят в режим испуга, резко меняют направление движения и перекрашиваются в синий цвет. За это время Пакман способен съесть призраков посредством столкновения с ними, которое безопасно. От съеденного привидения остаются только глаза, которые возвращаются в центр лабиринта, где призрак вновь оживает и отправляется в погоню за Пакманом. За съедение первого призрака после получения энерджайзера даётся 200 очков. За съедение каждого следующего привидения при эффекте того же энерджайзера даётся в два раза больше: 400, 800 и 1600 соответственно. Таким образом, при съедении всех призраков после каждого эффекта энерджайзера игрок может заработать за один уровень 12 000 очков.

\subsection{\label{subsec:ch01/sec04/subsec03}Поведение призраков.}
\begin{itemize}
	\item Красный призрак Шэдоу по прозвищу Блинки. В режиме преследования использует как цель ту клетку, в которой находится Пакман. Блинки, в отличие от других привидений, увеличивает свою скорость преследования относительно первоначальной дважды за уровень в зависимости от количества съеденных точек. Если точек осталось мало, то он меняет целевую клетку в режиме рассеивания на квадрат, в котором находится Пакман, и так гоняется за героем в двух режимах.
	\item Розовый призрак Спиди по прозвищу Пинки. В качестве цели при преследовании использует точку, находящуюся на четыре клетки впереди Пакмана. Однако из-за ошибки переполнения при движении Пакмана вверх Пинки использует в качестве цели квадрат, находящийся на четыре клетки вверх и на четыре влево от Пакмана.
	\item Голубой призрак Башфул по прозвищу Инки. Использует самый сложный алгоритм преследования: для определения направления движения строится отрезок, один из концов которого определяется положением Блинки, а середина находится на 2 клетки перед Пакманом. Второй конец отрезка --- искомая целевая точка. Получившуюся точку сложно предсказать, поэтому Инки считается самым опасным привидением. Из-за ошибки переполнения, аналогичной в поведении Пинки, во время движения Пакмана вверх целевая клетка Инки это две клетки вверх и две влево от Пакмана.
	\item Оранжевый призрак Поки по прозвищу Клайд. Если Клайд находится дальше 8 клеток от Пакмана, то он использует в качестве цели самого Пакмана, как Блинки. Если же Пакман ближе 8 клеток, то Клайд стремится к левому нижнему углу, как при рассеивании.
\end{itemize}


