\chapter{\label{ch:ch02}ГЛАВА 2. ПРАКТИЧЕСКАЯ ЧАСТЬ. ЭТАПЫ РАЗРАБОТКИ ИГРЫ. ТЕСТИРОВАНИЕ ПРОГРАММЫ.}

\section{\label{sec:ch02/sec01}Установка частоты кадров игрового поля.}
\textbf{FPS} --- частота кадров в секунду, демонстрируемая игрой.

Для установки FPS можно использовать стандартные методы библиотеки SDL2 для C/C++:~\url{https://lazyfoo.net/tutorials/SDL/} (см. разделы \texttt{Timing, Advanced Timers, Calculating Frame Rate, Capping Frame Rate)}.

Конечно, данные методы можно реализовать в языке программирования Python, используя библиотеку PySDL2. Вот примеры пользователей с различных форумов:~\url{https://www.cyberforum.ru/python-pygame/thread3086932.html}, \url{https://www.cyberforum.ru/blogs/416874/blog6407-page2.html#comment31430} (в данном примере, пльзователь реализует тот же самый метод, только использует библиотеку time).

Также для PySDL2 существует надстройка, написаннная при помощи модуля Cython. Она реализует конвертацию библиотеки pygame в PySDL2. В этом случае непонятно: это будет надстройка, которая будет жить в <<симбиозе>> с библиотекой PySDL2, или же это будет выступать в роли отдельной библиотеки?~\url{https://lukems.github.io/py-sdl2/modules/sdl2ext_time.html} (на сайте также есть ссылка на гит этой надстройки <<PyGameSDL2’s time>>).
